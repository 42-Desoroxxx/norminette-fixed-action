\documentclass{42-es}

%******************************************************************************%
%                                                                              %
%                               Prologue                                       %
%                                                                              %
%******************************************************************************%

\begin{document}
\title{La Norma}
\subtitle{Version 4}

\summary
{
      Este documento describe la norma aplicable en 42. Una norma de
      programación
      define un conjunto de reglas a seguir al escribir código. La Norma se
      aplica
      a todos los proyectos de C dentro del Common Core por defecto, y a
      cualquier
      proyecto donde se especifique.
      This document describes the applicable standard (Norm) at 42. A
      programming standard defines a set of rules to follow when writing code.
      The Norm applies to all C projects within the Common Core by default, and
      to any project where it's specified.
}

\maketitle

\tableofcontents

%******************************************************************************%
%                                                                              %
%                                 Foreword                                     %
%                                                                              %
%******************************************************************************%
\chapter{Prefacio}

La \texttt{norminette} está escrita en python y es de código abierto. \\
Su repositorio está disponible en \url{https://github.com/42School/norminette}.\\
¡Pull request, sugerencias e issues serán bien recibidos!

\newpage

%******************************************************************************%
%
%                                   Pedago explanations                        %
%
%******************************************************************************%
\chapter{¿Por qué?}

La Norma ha sido cuidadosamente elaborada para cumplir con muchas
necesidades pedagógicas. Aquí están las razones más importantes para todas las
elecciones a continuación:
\begin{itemize}

      \item Secuenciación: programar implica dividir una tarea grande y
            compleja en una larga serie de instrucciones elementales. Todas
            estas instrucciones se ejecutarán en secuencia: una tras otra. Un
            principiante que comienza a crear software necesita una
            arquitectura
            simple y clara para su proyecto, con una comprensión completa de
            todas las instrucciones individuales y el orden preciso de
            ejecución. Los lenguajes de programación crípticos que hacen
            múltiples instrucciones aparentemente al mismo tiempo son
            confusos, las funciones que intentan abordar múltiples tareas
            mezcladas en la misma porción de código son fuente de errores.\\
            La Norma te pide que crees piezas de código simples, donde la
            tarea única de cada pieza pueda ser claramente entendida y
            verificada, y donde la secuencia de todas las instrucciones
            ejecutadas no deje lugar a dudas. Por eso pedimos un máximo de 25
            líneas en las funciones, también por qué se prohíben los
            \texttt{for}, \texttt{do .. while}, o ternarios.
      \item Aspecto: al intercambiar con tus amigos y compañeros de trabajo
            durante el proceso normal de aprendizaje entre pares, y también
            durante las evaluaciones entre pares, no quieres perder tiempo
            descifrando su código, sino hablar directamente sobre la lógica de
            la pieza de código.\\
            La Norma te pide que uses un aspecto específico, proporcionando
            instrucciones para el nombre de las funciones y variables,
            indentación, reglas de llaves, tabulaciones y espacios en muchos
            lugares... . Esto te permitirá echar un vistazo a otros códigos que
            te resultarán familiares y llegar directamente al punto en lugar de
            perder tiempo leyendo el código antes de entenderlo. La Norma
            también
            funciona como una marca registrada. Como parte de la comunidad 42,
            podrás reconocer el código escrito por otro estudiante o exalumno
            de
            42 cuando estés en el mercado laboral.

      \item Visión a largo plazo: hacer el esfuerzo de escribir código
            comprensible es la mejor manera de mantenerlo. Cada vez que alguien
            más, incluyéndote a ti, tenga que corregir un error o agregar una
            nueva característica, no tendrá que perder su valioso tiempo
            tratando de entender lo que hace si previamente hiciste las cosas
            de
            la manera correcta. Esto evitará situaciones en las que los
            fragmentos de código dejen de ser mantenidos solo porque lleva
            tiempo, y eso puede marcar la diferencia cuando hablamos de tener
            un
            producto exitoso en el mercado. Cuanto antes aprendas a hacerlo,
            mejor.

      \item Referencias: puedes pensar que algunas, o todas, las reglas
            incluidas
            en la Norma son arbitrarias, pero en realidad pensamos y leímos
            sobre
            qué hacer y cómo hacerlo. Te animamos encarecidamente a buscar por
            qué
            las funciones deben ser cortas y hacer una sola cosa, por qué el
            nombre de las variables debe tener sentido, por qué las líneas no
            deben ser más largas de 80 columnas, por qué una función no debe
            tener muchos parámetros, por qué los comentarios deben ser útiles,
            etc, etc, etc...

\end{itemize}

\newpage

%******************************************************************************%
%                                                                              %
%                                The Norm                                      %
%                                                                              %
%******************************************************************************%
\chapter{La Norma}

%******************************************************************************%
%                             Naming conventions                               %
%******************************************************************************%
\section{Denominación}

\begin{itemize}

      \item El nombre de una estructura debe comenzar con \texttt{s\_}.
      \item El nombre de un typedef debe comenzar con \texttt{t\_}.
      \item El nombre de un union debe comenzar con \texttt{u\_}.
      \item El nombre de un enum debe comenzar con \texttt{e\_}.
      \item El nombre de una variable global debe comenzar con \texttt{g\_}.
      \item Los nombres de variables y funciones solo pueden contener
            minúsculas, dígitos y '\_' (snake\_case).
      \item Los nombres de archivos y directorios solo pueden contener
            minúsculas, dígitos y '\_' (snake\_case).
      \item Los caracteres que no forman parte de la tabla ASCII estándar están
            prohibidos.
      \item Las variables, funciones y cualquier otro identificador deben usar
            snake case. Sin letras mayúsculas, y cada palabra separada por un
            guión bajo.
      \item Todos los identificadores (funciones, macros, tipos, variables,
            etc.)
            deben estar en inglés.
      \item Los objetos (variables, funciones, macros, tipos, archivos o
            directorios) deben tener nombres lo más explícitos o mnemótecnicos
            posibles.
      \item El uso de variables globales que no están marcadas como const y
            static está prohibido y se considera un error de norma, a menos que
            el proyecto las permita explícitamente.
      \item El archivo debe compilar. Un archivo que no compila no se espera
            que
            pase la Norma.
\end{itemize}
\newpage

%******************************************************************************%
%                                 Formatting                                   %
%******************************************************************************%
\section{Formato}

\begin{itemize}

      \item Debes indentar tu código con tabulaciones de 4 espacios.
            Esto no es lo mismo que 4 espacios normales, estamos hablando de
            verdaderas tabulaciones.
      \item Cada función debe tener un máximo de 25 líneas, sin contar las
            llaves
            de la función.
      \item Cada línea debe tener como máximo 80 columnas de ancho, incluidos
            los
            comentarios. Advertencia: una tabulación no cuenta como una
            columna,
            sino como
            el número de espacios que representa.
      \item Cada función debe estar separada por una nueva línea. Cualquier
            comentario
            o instrucción de preprocesador puede estar justo encima de la
            función. La
            nueva línea está después de la función anterior.
      \item Una instrucción por línea.
      \item Una línea vacía debe estar vacía: sin espacios o tabulaciones.
      \item Una línea no puede terminar con espacios o tabulaciones.
      \item No puedes tener dos espacios consecutivos.
      \item Debes comenzar una nueva línea después de cada llave de apertura o
            cierre o después de una estructura de control.
      \item Salvo que sea el final de una linea, cada coma o punto y coma
            debe ser seguido por un espacio.
      \item Cada operador u operando debe estar separado por un (y solo un)
            espacio.
      \item Cada palabra clave de C debe ir seguida de un espacio, excepto las
            palabras clave para tipos (como int, char, float, etc.), así como
            sizeof.
      \item Cada declaración de variable debe estar indentada en la misma
            columna
            dentro de su scope.
      \item Los asteriscos que acompañan a los punteros deben estar pegados a
            los
            nombres de las variables.
      \item Una sola declaración de variable por línea.
      \item La declaración y la inicialización no pueden estar en la misma
            línea,
            excepto para las variables globales (cuando se permiten), variables
            estáticas y
            constantes.
      \item Las declaraciones deben estar al principio de una función.
      \item En una función, debes colocar una línea vacía entre las
            declaraciones
            de variables y el resto de la función. No se permiten otras líneas
            vacías en una función.
      \item Las asignaciones múltiples están completamente prohibidas.
      \item Puedes agregar una nueva línea después de una instrucción o
            estructura
            de control, pero tendrás que agregar una indentación con llaves o
            operador de asignación. Los operadores deben estar al principio de
            una línea.
      \item Las estructuras de control (if, while, etc...) deben tener llaves,
            salvo que contengan una sola línea.
      \item Las llaves que siguen a las funciones, declaradores o estructuras
            de
            control deben estar precedidas y seguidas por una nueva línea.

\end{itemize}

Ejemplo general de formato:
\begin{42ccode}
      int	g_global;
      typedef struct	s_struct
      {
      char    *my_string;
      int     i;
      } 	t_struct;
      struct		s_other_struct;

      int main(void)
      {
                  int	  i;
                  char	  c;

                  return (i);
            }
\end{42ccode}
\newpage

%******************************************************************************%
%                              Function parameters                             %
%******************************************************************************%
\section{Funciones}

\begin{itemize}

      \item Una función no puede recibir más de 4 parámetros.
      \item Una función que no recibe argumentos debe ser prototipada con la
            palabra ``void'' como argumento.
      \item Los parámetros en los prototipos de las funciones deben tener
            nombre.
      \item Cada función debe estar separada de la siguiente por una línea
            vacía.
      \item No puedes declarar más de 5 variables por función.
      \item El retorno de una función debe estar entre paréntesis.
      \item Cada función debe tener una sola tabulación entre su tipo de
            retorno
            y
            su nombre.

            \begin{42ccode}
                  int my_func(int arg1, char arg2, char *arg3)
                  {
                              return (my_val);
                        }

                  int func2(void)
                  {
                              return ;
                        }
            \end{42ccode}
\end{itemize}
\newpage

%******************************************************************************%
%                        Typedef, struct, enum and union                       %
%******************************************************************************%
\section{Typedef, struct, enum y union}

\begin{itemize}
      \item Agrega una tabulación al declarar un struct, enum o union.
      \item Al declarar una variable de tipo struct, enum o union, agrega un
            solo
            espacio en el tipo.
      \item Al declarar un struct, union o enum con un typedef, se aplican
            todas
            las reglas de indentación.
      \item El nombre del typedef debe ir precedido por una tabulación.
      \item Debes indentar todos los nombres de las estructuras en la misma
            columna dentro de su scope.
      \item No puedes declarar una estructura en un archivo .c.
\end{itemize}
\newpage

%******************************************************************************%
%                                   Headers                                    %
%******************************************************************************%
\section{Headers - a.k.a archivos include}

\begin{itemize}
      \item Las cosas permitidas en los archivos de cabecera son:
            inclusiones de cabecera (de sistema o no), declaraciones, defines,
            prototipos y macros.
      \item Todos los includes deben estar al principio del archivo.
      \item No puedes incluir un archivo C.
      \item Los archivos de cabecera deben estar protegidos de inclusiones
            dobles. Si el archivo es \texttt{ft\_foo.h}, su macro de protección
            es
            \texttt{FT\_FOO\_H}.
      \item Las inclusiones de cabecera no utilizadas (.h) están prohibidas.
      \item Todas las inclusiones de cabecera deben estar justificadas en un
            archivo .c, así como en un archivo .h.
\end{itemize}

\begin{42ccode}
      #ifndef FT_HEADER_H
      # define FT_HEADER_H
      # include <stdlib.h>
      # include <stdio.h>
      # define FOO "bar"

      int	g_variable;
      struct	s_struct;

      #endif
\end{42ccode}
\newpage

%******************************************************************************%
%                                 The 42 header                                %
%******************************************************************************%

\section{La cabecera de 42 - a.k.a cómo empezar un archivo con estilo}

\begin{itemize}
      \item Todos los archivos .c y .h deben comenzar inmediatamente con la
            cabecera 42 estándar: un comentario de varias líneas con un formato
            especial
            que incluye información útil. La cabecera estándar está
            naturalmente
            disponible
            en las computadoras en clusters para varios editores de texto
            (emacs:
            usando
            \texttt{C-c C-h}, vim usando \texttt{:Stdheader} o \texttt{F1},
            etc...).
      \item La cabecera 42 debe contener varias informaciones actualizadas,
            incluyendo
            el creador con login y correo electrónico, la fecha de creación, el
            login y
            la fecha de la última actualización. Cada vez que el archivo se
            guarda en
            disco, la información debe actualizarse automáticamente.
\end{itemize}
\newpage

%******************************************************************************%
%                           Macros and Pre-processors                          %
%******************************************************************************%
\section{Macros y Preprocesadores}

\begin{itemize}

      \item Las constantes del preprocesador (o \#define) que crees, deben
            usarse solo para valores literales y constantes.
      \item Todos los \#define creados para evitar la norma y/o para ofuscar
            el código están prohibidos. Esta parte debe ser verificada por un
            humano.
      \item Puedes usar macros disponibles en bibliotecas estándar, solo si
            esas están permitidas en el alcance del proyecto dado.
      \item Las macros de varias líneas están prohibidas.
      \item Los nombres de las macros deben estar en mayúsculas.
      \item Debes indentar los caracteres que siguen a \#if, \#ifdef o
            \#ifndef.
      \item Las instrucciones del preprocesador están prohibidas fuera del
            alcance global.
\end{itemize}
\newpage

%******************************************************************************%
%                              Forbidden stuff!                                %
%******************************************************************************%
\section{¡Cosas Prohibidas!}

\begin{itemize}
      \item No puedes utilizar:
            \begin{itemize}

                  \item for
                  \item do...while
                  \item switch
                  \item case
                  \item goto

            \end{itemize}
      \item Operadores ternarios como `?'.
      \item VLAs - o arrays de longitud variable.
      \item Tipos implícitos en declaraciones de variables.

\end{itemize}
\begin{42ccode}
      int main(int argc, char **argv)
      {
                  int i;
                  char	string[argc]; // This is a VLA

                  i = argc > 5 ? 0 : 1 // Ternary
            }
\end{42ccode}
\newpage

%******************************************************************************%
%                                   Comments                                   %
%******************************************************************************%
\section{Comentarios}

\begin{itemize}
      \item Los comentarios no pueden estar dentro de los cuerpos de las
            funciones. Los comentarios deben estar al final de una línea, o en
            su propia
            línea.
      \item Tus comentarios deben estar en inglés. Y deben ser útiles.
      \item Un comentario no puede justificar la creación de un carryall o
            una mala función.
\end{itemize}

\warn{
      Un carryall o mala función generalmente viene con nombres que no son
      explícitos como f1, f2... para la función y a, b, i,.. para las
      declaraciones.
      Una función cuyo único objetivo es evitar la norma, sin un propósito
      lógico único, también se considera como una mala función.
      Por favor, recuerda que es preferible tener funciones claras y legibles
      que logren una tarea clara y simple cada una. Evita cualquier técnica
      de obfuscación de código, como la de una sola línea.
}
\newpage

%******************************************************************************%
%                                    Files                                     %
%******************************************************************************%
\section{Archivos}

\begin{itemize}
      \item No puedes incluir un archivo .c
      \item No puedes tener más de 5 definiciones de funciones en un archivo
            .c.
\end{itemize}
\newpage

%******************************************************************************%
%                                   Makefile                                   %
%******************************************************************************%
\section{Makefile}
Los Makefiles no son verificados por la Norma, y deben ser verificados durante
la evaluación por el estudiante.

\begin{itemize}

      \item El \$(NAME), clean, fclean, re y all son reglas obligatorias.
      \item Si el makefile hace relinks, el proyecto se considerará no funcional.
      \item En el caso de un proyecto multibinario, además de las reglas
            anteriores, debes tener una regla que compile ambos binarios, así
            como una regla específica para cada binario compilado.
      \item En el caso de un proyecto que llama a una función de una librería no del sistema (por ejemplo: libft), tu makefile debe compilar esta biblioteca automáticamente.
      \item Todos los archivos fuente que necesitas para compilar tu proyecto deben estar nombrados explícitamente en tu Makefile.
\end{itemize}

\end{document}
%******************************************************************************%
