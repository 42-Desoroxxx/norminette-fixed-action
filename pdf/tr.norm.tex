% vim: set ts=4 sw=4 tw=80 noexpandtab:
\documentclass{42-en}
\newcommand\qdsh{\texttt{42sh}}

%******************************************************************************%
%                                                                              %
%                               Prologue                                       %
%                                                                              %
%******************************************************************************%

\begin{document}
\title{Norm}
\subtitle{Sürüm3}

\summary
{
    Özet: Bu belge 42’de uygulanabilecek standardı (Norm) tanımlamaktadır.
    Bir programlama standardı, kod yazarken uyulması gereken kurallar bütününü
    tanımlar. Norm, aksi kararlaştırılmadıkça Inner Circle kapsamındaki
    tüm C projelerine, özel olarak belirtildiği takdirde de diğer her türlü
    projeye uygulanır.
}

\maketitle

\tableofcontents



%******************************************************************************%
%                                                                              %
%                                 Foreword                                     %
%                                                                              %
%******************************************************************************%
\chapter{Önsöz}

    Norm, Python’da yazılmıştır ve açık kaynaklı bir projedir(?).
    Veri havuzuna https://github.com/42School/norminette üzerinden erişilebilir.
    Pull request, öneri ve sorunlara ilişkin iletişime geçmekten çekinmeyin!

%******************************************************************************%
%                                                                              %
%                                The Norm                                      %
%                                                                              %
%******************************************************************************%
\chapter{Norm}


%******************************************************************************%
%                             Naming conventions                               %
%******************************************************************************%
    \section{İsimlendirme}

        \begin{itemize}

            \item Bir struct’ın ismi mutlaka \texttt{s\_} ile başlamalıdır.

            \item Bir typedef’in ismi mutlaka \texttt{t\_} ile başlamalıdır

            \item Bir union’ın ismi mutlaka \texttt{u\_} ile başlamalıdır.

            \item Bir enum’ın ismi mutlaka \texttt{e\_} ile başlamalıdır.

            \item Bir global’ın ismi mutlaka \texttt{g\_} ile başlamalıdır.

            \item Değişkenler ile fonksiyonların isimleri yalnızca küçük harfler, rakamlar ve '\_' (Unix Case) içerebilir.

            \item Dosyalar ile dizinlerin isimleri yalnızca küçük harfler, rakamlar ve '\_' (Unix Case) içerebilir.

            \item Standart ASCII tablosunda yer almayan karakterlerin kullanımı yasaktır.

            \item Değişkenler, fonksiyonlar ve diğer tüm belirleyicilerde snake case kullanılmalıdır. Büyük harf kullanılmamalı ve her bir kelime alt çizgi ile ayrılmalıdır.

            \item Tüm belirleyiciler (fonksiyonlar, makrolar, tipler, değişkenler vs.) İngilizce olmalıdır.

            \item Nesneler (değişkenler, fonksiyonlar, makrolar, tipler, dosyalar veya dizinler) mümkün olan en açık ve
akılda kalıcı şekilde isimlendirilmelidir.

            \item Const ve static olarak işaretlenmemiş global değişlenlerin kullanılması yasaktır ve bu durum, projenin
bunlara açıkça izin vermediği hallerde norm hatası olarak değerlendirilir

            \item Bir dosya mutlaka derlenebilmelidir. Derlenemeyen bir dosyanın Norm’a uymasısöz konusu olmayacaktır.

        \end{itemize}
\newpage

%******************************************************************************%
%                                 Formatting                                   %
%******************************************************************************%
    \section{Format}

            \begin{itemize}

                \item Kodunuzu mutlaka 4 boşluk ile indentlemelisiniz. Burada bahsedilen boşluk 4 ortalama boşluk anlamına
gelmemekte olup, gerçek anlamda tab tuşuna basılmasını ifade etmektedir.

                \item Her fonksiyon, fonksiyonun kendi kıvrımlı ayraçları (curly bracket) hariç, maksimum 25 satırdan oluşmalıdır.

                \item Her satır, yorumlarla birlikte, maksimum 80 sütun genişliğinde olmalıdır. Uyarı: bir kez tablanmış olma bir
sütun olarak sayılmamakta, karşılık geldiği boşluk sayısı kadar dikkate alınmaktadır.

                \item Her fonksiyon yeni bir satır başı ile ayrılmalıdır. Herhangi bir yorum veya ön işlemci, fonksiyonun hemen
üzerinde yer alabilir. Satır başı bir önceki fonksiyondan sonra gelir.

                \item Her satırda tek bir talimat yer almalıdır.

                \item Boş bir satır mutlaka boş olmalıdır: herhangi bir boşluk veya tab olmamalıdır.

                \item Bir satır asla boşluk veya tab ile bitemez.

                \item Hiçbir zaman peş peşe iki boşluk bırakamazsınız.

                \item Her bir kıvrımlı ayraçtan (curly bracket) sonra veya kontrol yapısının sonunda yeni bir satıra geçmelisiniz.

                \item Bir satırın sonu olmadığı takdirde, her virgül ve noktalı virgülden sonra bir boşluk bırakılmalıdır.

                \item Her bir operatör veya operand yalnızca ve yalnızca tek bir boşluk ile ayrılmalıdır.

                \item Tipler (int, char, float vs. gibi) için olanlar hariç her bir C sözcüğünden (keyword) ve sizeof’tan sonra bir
boşluk bırakılmalıdır.

                \item Her bir değişken declarationı kendi kapsamına göre ilgili sütunda indentlenmiş olmalıdır

                \item Pointer’larla birlikte kullanılan asteriskler değişken isimlerine bağlı olmalıdır

                \item Her satırda tek bir değişken declarationı yer almalıdır.

                \item Global değişkenler (izin verilmesi halinde), statik değişkenler ve sabitler dışında, declarationlar ve initializationlar aynı satırda yer alamaz.

                \item Declarationlar fonksiyonların başında yer almalıdır.

                \item Bir fonksiyonda, değişken declarationları ile fonksiyonun geri kalanı arasında boş bir satır bırakmalısınız.
Fonksiyonda başka herhangi bir boş satıra izin verilmez.

                \item Çoklu atamalar kesinlikle yasaktır.

                \item Bir talimat ya da structuredan sonra yeni bir satır ekleyebilirsiniz, ancak bu durumda ayraçlar (brackets)
veya atama operatörü ile bir girinti (indentation) eklemeniz gerekecektir. Operatörler satırın başında olmalıdır.

                \item Kontrol yapılarında (if, while..), tek bir çizgi ya da tek bir koşul içerdikleri haller dışında, ayraç (brace)
bulunmalıdır.

            \end{itemize}

            \newpage

            General example:
            \begin{42ccode}
int             g_global;
typedef struct  s_struct
{
    char    *my_string;
    int     i;
}               t_struct;
struct          s_other_struct;

int     main(void)
{
    int     i;
    char    c;

    return (i);
}
            \end{42ccode}
            \newpage

%******************************************************************************%
%                              Function parameters                             %
%******************************************************************************%
    \section{Fonksiyonlar}

        \begin{itemize}

            \item Bir fonksiyon maksimum 4 isimlendirilmiş parametre alabilir.

            \item Herhangi bir parametre almamış fonksiyonlar mutlaka parametre kısmına açıkça ‘void ’yazılarak prototiplendirilmelidir.

            \item Fonksiyonların prototiplerindeki parametreler mutlaka isimlendirilmelidir.

            \item Her fonksiyon bir sonrakinden mutlaka boş bir satır ile ayrılmalıdır.

            \item Fonksiyon başına 5 değişkenden fazlasını declare edemezsiniz.

            \item Bir fonksiyonun geri dönüşü (return) parantez içinde olmalıdır. 

            \item Her fonksiyonun geri dönüş (return) tipi ve ismi arasında tek bir tab bulunmalıdır.

            \begin{42ccode}
int my_func(int arg1, char arg2, char *arg3)
{
    return (my_val);
}

int func2(void)
{
    return ;
}
            \end{42ccode}

        \end{itemize}
        \newpage


%******************************************************************************%
%                        Typedef, struct, enum and union                       %
%******************************************************************************%
    \section{Typedef, struct, enum ve union}

        \begin{itemize}

            \item Bir struct, enum veya union declare edilirken bir tab ekleyin.

            \item Bir tip struct, enum veya unionının değişkenini declare ederken tipe tek bir boşluk ekleyin.

            \item Typedef ile bir struct, union veya enum declare ederken, tüm indentleme kuralları uygulanır. Typedef’in adı
ile struct/union/enum’ın adını hizalamalısınız.

            \item Tüm structureların isimlerini kendi kapsamına göre ilgili sütunda indentlemelisiniz.

            \item Bir structureı bir .c dosyasının içinde declare edemezsiniz.

        \end{itemize}
        \newpage


%******************************************************************************%
%                                   Headers                                    %
%******************************************************************************%
    \section{Header Dosyaları}

        \begin{itemize}

            \item Başlık dosyalarında izin verilenler: başlık inclusionları (sistem veya değil), declarationlar, prototipler ve
makrolar.

            \item Tüm includelar dosyanın başlangıcında olmalıdır

            \item Bir C dosyası include edemezsiniz.

            \item Başlık dosyaları çift inclusiondan korunmalıdır. Eğer dosya \texttt{ft\_foo.h} ise, onun makro karşılığı \texttt{FT\_FOO\_H}
şeklindedir.

            \item Kullanılmayan başlık inclusionları (.h) yasaktır

            \item Tüm başlık inclusionları hem .c dosyasında hem de .h dosyasında doğrulanmalıdır.

        \end{itemize}

        \begin{42ccode}
#ifndef FT_HEADER_H
# define FT_HEADER_H
# include <stdlib.h>
# include <stdio.h>
# define FOO "bar"

int		g_variable;
struct	s_struct;

#endif
        \end{42ccode}
        \newpage

%******************************************************************************%
%                           Macros and Pre-processors                          %
%******************************************************************************%
    \section{Makrolar ve Ön İşlemciler}

        \begin{itemize}

            \item Yarattığınız ön işlemci sabitleri (veya \#define) yalnızca gerçek ve sabit değerler için kullanılmalıdır.

            \item Normu bypass etmek ve/veya kod karıştırmak için yaratılan tüm \#define yasaktır. Bu kısım bir insan tarafından kontrol edilmelidir.

            \item Standart kütüphanelerdeki makroları, yalnızca verilen projelerin kapsamında bunlara izin verilmesi halinde
kullanabilirsiniz.

            \item Multiline makrolar yasaktır.

            \item Makro isimlerinin tamamı büyük harf olmalıdır.

            \item \#if, \#ifdef veya \#ifndefden sonra gelen karakterleri indentlemelisiniz.

        \end{itemize}
        \newpage


%******************************************************************************%
%                              Forbidden stuff!                                %
%******************************************************************************%
    \section{Yasaklar!}

        \begin{itemize}

            \item Aşağıdakileri kullanma izniniz bulunmamaktadır:

                \begin{itemize}

                    \item for
                    \item do...while
                    \item switch
                    \item case
                    \item goto

                \end{itemize}

            \item ‘? ’gibi ternary operatörleri.

            \item VLAlar - Variable Length Arrays (Değişken Uzunluklu Dizi)

            \item Değişlen declarationlarında implicit tip.

        \end{itemize}
        \begin{42ccode}
    int main(int argc, char **argv)
    {
        int     i;
        char    string[argc]; // This is a VLA

        i = argc > 5 ? 0 : 1 // Ternary
    }
        \end{42ccode}
        \newpage

%******************************************************************************%
%                                   Comments                                   %
%******************************************************************************%
    \section{Yorumlar}

        \begin{itemize}

            \item Yorumlar fonksiyon gövdelerinin içinde olamaz. Yorumlar bir satırın veya kendi satırlarının sonunda olmalıdır.

            \item Yorumlarınız İngilizce ve kullanışlı olmalıdır.

            \item Bir yorum ‘nahoş’ bir fonksiyonu doğrulayamaz.

        \end{itemize}
        \newpage


%******************************************************************************%
%                                    Files                                     %
%******************************************************************************%
    \section{Dosyalar}

        \begin{itemize}

            \item Bir .c dosyasını include edemezsiniz.

            \item Bir .c dosyasında 5 taneden fazla fonksiyon tanımına yer veremezsiniz.

        \end{itemize}
        \newpage


%******************************************************************************%
%                                   Makefile                                   %
%******************************************************************************%
    \section{Makefile}

            Makefilelar Norm tarafından kontrol edilmez, öğrenci tarafından gelişim süresince kontrol edilmelidir.

            \begin{itemize}

                \item \$(NAME), clean, fclean, re ve tüm kurallar bağlayıcıdır.

                \item Makefile relink ederse, proje işlevsiz kabul edilecektir

                \item Çoklu ikili bir proje söz konusu ise, yukarıdaki kurallara ek olarak, her iki ikiliyi de compile eden bir kural
ile compile edilen her bir ikiliye özgü ayrı bir kurala sahip olmalısınız.

                \item Sistem dışı bir kütüphaneden (libft vs.) fonksiyon çağıran bir proje söz konusu ise, makefileınız bu kütüphane
ile otomatik olarak compile etmelidir.

                \item Projenizi compile etmeniz gereken tüm kaynak dosyalar Makefile’ınızda açıkça isimlendirilmiş olmalıdır.

            \end{itemize}

   \newpage
%******************************************************************************%
%                                   Pedago explanations                        %
%******************************************************************************%
    \section{Neden?}

    Norm, pek çok pedagojik ihtiyacı karşılamak amacıyla titizlikle hazırlanmıştır. Yukarıdaki tüm seçenekler için
en önemli nedenler şu şekildedir:

    \begin{itemize}

    \item Sıralama: Kodlama büyük ve karmaşık bir görevin uzun bir dizi basit
        talimatlara bölünmesini sağlar. Tüm bu talimatlar, biri diğerini izleyecek
        şekilde, sırayla yerine getirilir. Yazılım yaratmaya yeni başlayan bir kişi,
        projesi için, tüm bireysel talimatlara ve gerçekleştirileceklerin doğru
        sıralanmasına yönelik tam bir anlayış ile birlikte basit ve yalın bir mimariye
        gereksinim duyar. Aynı anda birden fazla talimatı yerine getiren kriptik dildeki
        dizilimler kafa karıştırıcıyken, tek bir kod paydasında karışık bir şekilde yer
        alan birden fazla göreve işaret eden fonksiyonlar ise hatalara kaynak teşkil eder.
        Norm sizden, her bir parçanın kendine özgü görevinin açıkça anlaşılabilir ve
        doğrulanabilir olduğu ve tüm talimatların uygulanmasına ilişkin sıralamanın
        şüpheye mahal vermeyecek nitelik arz ettiği basit kod paydaları oluşturmanızı
        talep eder. Fonksiyonlarda maksimum 25 satıra yer verilmesini talep etmemizin ve for,
        do…while veya ternary kullanımlarının yasak olmasının sebebi de budur.

    \item Görünüm ve Tavır: Arkadaşlarınızla bilgi alışverişi esnasında veya sizinle
        aynı pozisyondaki iş arkadaşlarınızla birlikte öğrenme ve ayrıca birbirinizi
        değerlendirme süreçlerinizde, onların kodlarını deşifre etmekle vakit kaybetmek
        istemez, doğrudan kodlarının arkasında yatan mantığa dair sohbet etmek istersiniz.
        Norm sizden, fonksiyon ve değişkenlerin isimlendirilmesi, indent edilmesi, ayraç
        (brace) kuralları, pek çok yerde kullanılan tab ve boşluklara vs. ilişkin kendinize
        özgü bir görünüm ve tavır kullanmanızı talep eder. Bu size, başkalarının sizinkilere
        benzer görünümdeki kodlarını, kodları anlamadan önce okumaya vakit harcamaya gerek
        kalmaksızın, kolaylıkla incelemeniz imkanını sunar. Norm aynı zamanda bir marka değeri
        taşımaktadır. 42 topluluğunun bir parçası olarak, iş pazarına girdiğinizde, diğer 42
        öğrenci veya mezunları tarafından yazılmış kodları da tanıyor olma imkanına sahip
        olacaksınız.

    \item Uzun vadeli vizyon: Anlaşılabilir bir kod yazmak için gerekli çabayı sarf etmek,
        onun sürdürülebilirliğini sağlamanın en iyi yoludur. Siz dahil herhangi bir kişinin,
        herhangi bir bugı onarmaya veya yeni bir özellik eklemeye ihtiyaç duyduğu her an,
        bir önceki kişi işini doğru şekilde yapmış ise, kıymetli vaktinizi kaybetmenizin
        önüne geçilmiş olacaktır. Bu da kodların zaman kaybı nedeniyle sürdürülebilirliğini
        yitirmesinin engellenmesini sağlayacak ve pazarda başarılı bir ürüne sahip olmaktan
        bahsederken fark yaratacaktır. Bunu yapmayı ne kadar erken öğrenirseniz, sizin için o
        kadar iyi olur.

    \item Norm’da yer alan kuralların bir kısmının veya tamamının tartışmaya açık olduğunu
        düşünebilirsiniz, ancak biz ne yapılması ve nasıl yapılması gerektiğine ilişkin çok
        fazla düşündük ve araştırma yaptık. Sizi fonksiyonların neden kısa olması ve tek bir
        şey yapmaya yönelik olması gerektiğine, değişkenlerin isimlerinin neden bir anlam
        ifade etmesi gerektiğine, satırların neden 80 sütun genişliğinden daha uzun olmaması
        gerektiğine, bir fonksiyonun neden birden fazla parametre almaması gerektiğine,
        yorumların neden faydalı olması gerektiğine vs. vs. ilişkin bir Google araması
        yapmaya teşvik etmek isteriz.

    \end{itemize}

\end{document}
%******************************************************************************%
